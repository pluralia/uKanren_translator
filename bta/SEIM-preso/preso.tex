%% Простая презентация с примером включения программного кода и
%% пошаговых спецэффектов
\documentclass{beamer}
\usetheme{SPbAU}
%\useoutertheme{infolines}
\usepackage{tikz}
\usepackage{fontspec}
\usepackage{xunicode}
\usepackage{xltxtra}
\usepackage{xecyr}
\usepackage{hyperref}
\setmainfont[Mapping=tex-text]{DejaVu Serif}
\setsansfont[Mapping=tex-text]{DejaVu Sans}
\setmonofont[Mapping=tex-text]{DejaVu Sans Mono}
\usepackage{polyglossia}
\setdefaultlanguage{russian}
\usepackage{graphicx}
\usepackage{listings}
\usepackage{stmaryrd}
\usepackage{mathtools}
\lstdefinestyle{mycode}{
  belowcaptionskip=1\baselineskip,
  breaklines=true,
  xleftmargin=\parindent,
  showstringspaces=false,
  basicstyle=\footnotesize\ttfamily,
  keywordstyle=\bfseries,
  commentstyle=\itshape\color{gray!40!black},
  stringstyle=\color{red},
  numbers=left,
  numbersep=5pt,
  numberstyle=\tiny\color{gray},
}
\lstset{escapechar=@,style=mycode,mathescape=true}

\newcommand{\miniKanren}{\textsc{miniKanren}}
\newcommand{\mercury}{\textsc{Mercury}}
\newcommand{\haskell}{\textsc{Haskell}}
\newcommand{\prolog}{\textsc{Prolog}}
\newcommand{\scheme}{\textsc{Scheme}}
\newcommand{\logen}{\textsc{LOGEN}}
\newcommand{\github}{\textsc{GitHub}}

\newcommand{\beginbackup}{
   \newcounter{framenumbervorappendix}
   \setcounter{framenumbervorappendix}{\value{framenumber}}
}
\newcommand{\backupend}{
   \addtocounter{framenumbervorappendix}{-\value{framenumber}}
   \addtocounter{framenumber}{\value{framenumbervorappendix}} 
}

%%%%%%%%%%%%%%%%%%%%%%%%%%%%%%%%%%%%%%%%%%%%%%%%%%%%%%%%%%%%%%%%%%%%%%%%%%%%%%%%

\title[Анализ времени связывания]{Анализ времени связывания для реляционных программ}
\institute[JetBrains Research]{
JetBrains Research, Университет ИТМО
}

\author[Артемьева И., Вербицкая Е.]{\textbf{Ирина Артемьева}, Екатерина Вербицкая}

\date{16.05.2020}

\begin{document}
{
\begin{frame}[fragile]
  \begin{tabular}{p{4.5cm} p{7cm}}
   \begin{center}
      \includegraphics[height=1.5cm]{pictures/jetbrainsResearch.pdf}
    \end{center}
    &
    \begin{center}
      \includegraphics[height=2.3cm]{pictures/itmo.png}
    \end{center}
  \end{tabular}
  \titlepage
\end{frame}
}

\begin{frame}\frametitle{Реляционное программирование}
\begin{itemize}
    \item Программы реляционного программирования = математические отношения
    \item Можно исполнять в различных направлениях: зафиксировав часть аргументов, находить остальные
    \item $append^o \subseteq [a] \times [a] \times [a]$ --- отношение, связывающее три списка; третий аргумент --- конкатенация первых двух ($?$ --- выходной):
        \begin{itemize}
            \item $append^o \ x \ y \ ?$ --- конкатенация $x$ и $y$
            \item $append^o \ ? \ y \ z$ --- разность $z$ и $y$
            \item $append^o \ ? \ ? \ z$ --- все пары списков, конкатенация которых равна $z$
            \item $append^o \ ? \ ? \ ?$ --- все списки отношения
        \end{itemize}
\end{itemize}
\end{frame}

\begin{frame}\frametitle{Преимущество реляционного программирования}
Выбор направления: написав одну программу, получаем несколько целевых функций
    \begin{itemize}
        \item по интерпретатору языка и набору тестов синтезируем программу
        \item по предикату "последовательность вершин в графе формирует желаемый путь" синтезируем генератор таких путей
    \end{itemize}
\end{frame}

\begin{frame}\frametitle{Язык \miniKanren{}}
\begin{itemize}
    \item Основной представитель парадигмы реляционного программирования
    \item Встраивается в языки общего назначения
    \item Все языковые конструкции обратимы
        \begin{itemize}
            \item в отличие от \prolog{}, который использует необратимую операцию cut, что делает невозможным выполнение по направлениям
        \end{itemize}
\end{itemize}
\end{frame}

\begin{frame}[fragile]\frametitle{Абстрактный синтаксис \miniKanren{}}
Конструкции на примере отношения $append^o$:
\begin{center}
\begin{lstlisting}
:: $append^o$ x y z =
  (x === [] /\ z === y) \/
  ([h t r:
    (x === h $\%$ t /\
     z === h $\%$ r /\
     {$append^o$ t y r}])
\end{lstlisting}
\end{center}
\end{frame}

\begin{frame}\frametitle{Проблема скорости вычисления}
\begin{itemize}
    \item При написании программы подразумевается конкретное направление, называемое прямым; все другие --- обратные
    \item Выполнение в обратном направлении обычно крайне неэффективно
\end{itemize}
\end{frame}

\begin{frame}\frametitle{Подходы к ускорению выполнения}
\begin{itemize}
    \item Специализация
        \begin{itemize}
            \item Конъюнктивная частичная дедукция (CPD)
                \begin{itemize}
                    \item из отношения удаляются избыточные вычисления, зависимые от известного входа
                \end{itemize}
            \item Суперкомпиляция
                \begin{itemize}
                    \item отношение исполняется с сохранением истории вычислений --- на её основе происходит оптимизация
                \end{itemize}
        \end{itemize}
    \item \textbf{\emph{Трансляция}}
        \begin{itemize}
            \item По отношению с фиксированным направлением генерируется функция на функциональном языке программирования        \end{itemize}
\end{itemize}
\end{frame}

\begin{frame}\frametitle{Трансляция}
Отношение $+$ Направление $\Mapsto$ Функция
\begin{itemize}
    \item Несколько выходных переменных объединятся в кортеж
    \item Вариативность результа выражается списком
    \item \textbf{\emph{Неопределённость порядка вычислений}}
\end{itemize}
\end{frame}

\lstset{language=haskell}
\begin{frame}[fragile]\frametitle{Неопределённость порядка вычислений}
\begin{itemize}
    \item Рассмотрим порядок вычислений 2 дизъюнкта $append^o$ в прямом и обратном направлении
\end{itemize}
\begin{lstlisting}
:: $append^o$ x y z =
  (x === [] /\ z === y) \/
  ([h t r:
    (x === h $\%$ t /\
     z === h $\%$ r /\
     {$append^o$ t y r}])
\end{lstlisting}
\begin{tabular}{p{5cm} p{5cm}}
    \begin{center}
        \begin{itemize}
            \item $(h~\%~t)~\coloneqq~x$
            \item $r~\coloneqq~append^o~t~y$
            \item $z~\coloneqq~(h~\%~r)$
            \item $return \ z$
        \end{itemize}
    \end{center}
&
    \begin{center}
        \begin{itemize}
            \item $(h~\%~r)~\coloneqq~z$
            \item $(t,~y)~\coloneqq~append^o~r$
            \item $x~\coloneqq~h~\%~t$
            \item $return \ (x, \ y)$
        \end{itemize}
    \end{center}
\end{tabular}
\begin{itemize}
    \item Решение --- \textbf{\emph{анализ времени связывания}}
\end{itemize}
\end{frame}

\begin{frame}\frametitle{Анализ времени связывания}
\begin{itemize}
    \item Вход: программа на \miniKanren{} и имена входных переменных
    \item Каждой переменной сопоставляется время её связывания
    \item Ранее не существовало для \miniKanren{}
\end{itemize}
\end{frame}

\begin{frame}\frametitle{Цель и задачи}
\begin{itemize}
    \item Цель: указать порядок, в котором имена переменных связываются со значениями
    \item Задача: разработать алгоритм анализа времени связывания для \miniKanren{}
\end{itemize}
\end{frame}

\begin{frame}\frametitle{Аналоги анализа времени связывания}
\begin{itemize}
    \item Для offline-специализации
    %\footnote{N. D. Jones, C. K. Gomard, and P. Sestoft, "Partial evaluation and automatic program generation"}
        \begin{itemize}
            \item какие данные известны статически и могут быть учтены при специализации
        \end{itemize}
    \item \mercury{} --- для эффективной компиляции
    %\footnote{W. Vanhoof, M. Bruynooghe and M. Leuschel, "Binding-time  analysis  for  mercury"}
        \begin{itemize}
            \item два типа аннотаций --- недостаточно
            \item использует граф типов --- в \miniKanren{} нет типов
        \end{itemize}
    \item \textbf{\emph{Лямбда-исчисление с функциями высшего порядка}}
    %\footnote{P. Thiemann, "Binding-time analysis for lambda calculus"}
        \begin{itemize}
            \item определяется порядок связывания переменных
            \item типы аннотаций --- $\{ 0, 1, \dots, N\}$
        \end{itemize}
\end{itemize}
\end{frame}

\begin{frame}\frametitle{Аннотирование}
\begin{itemize}
    \item Время связывания переменной --- число или $Undef$ (время связывания неизвестно)
    \item Процесс подбора чисел назовем \textit{\emth{аннотированием}}
    \item Изначально входные переменные аннотируются $0$, остальные --- $Undef$ 
    \item Аннотация никогда не заменяется на меньшую
\end{itemize}
\end{frame}

\begin{frame}[fragile]\frametitle{Пример аннотирования $append^o$}
\begin{itemize}
    \item Число над переменной --- её аннотация
    \item Рассмотрим прямое и обратное направление
\end{itemize}
\begin{tabular}{p{5.5cm} p{7cm}}
    \begin{center}
    \begin{lstlisting}
:: $append^o$ $x^0$ $y^0$ $z^1$ = 
  ($x^0$ === [] /\
   $y^0$ === $z^1$) \/
  ([h, t, r:
     $x^0$ === $h^1$ $\%$ $t^1$ /\
     $z^3$ === $h^1$ $\%$ $r^2$ /\
     {$append^o$ $t^1$ $y^0$ $r^2$}])
    \end{lstlisting}
    \end{center}
&
    \begin{center}
    \begin{lstlisting}
:: $append^o$ $x^1$ $y^1$ $z^0$ = 
  ($x^1$ === [] /\
   $y^1$ === $z^0$) \/
  ([h, t, r:
     $x^3$ === $h^1$ $\%$ $t^2$ /\
     $z^0$ === $h^1$ $\%$ $r^1$ /\
     {$append^o$ $t^2$ $y^2$ $r^1$}])
    \end{lstlisting}
    \end{center}
\end{tabular}
\begin{tabular}{p{5cm} p{5cm}}
    \begin{center}
        \begin{itemize}
            \item $(h~\%~t)~\coloneqq~x$
            \item $r~\coloneqq~append^o~t~y$
            \item $z~\coloneqq~(h~\%~r)$
            \item $return \ z$
        \end{itemize}
    \end{center}
&
    \begin{center}
        \begin{itemize}
            \item $(h~\%~r)~\coloneqq~z$
            \item $(t,~y)~\coloneqq~append^o~r$
            \item $x~\coloneqq~h~\%~t$
            \item $return \ (x, \ y)$
        \end{itemize}
    \end{center}
\end{tabular}
\end{frame}

\begin{frame}\frametitle{Алгоритм аннотирования}
\begin{itemize}
    \item Аннотировать отношение = аннотировать все его дизъюнкты
    \item Аннотировать дизъюнкт = аннотировать все его конъюнкты
        \begin{itemize}
            \item Переменная в конъюнктах одного дизъюнкта имеет одну аннотацию --- согласованность аннотаций
        \end{itemize}
    \item Аннотировать конъюнкт = аннотировать унификацию или вызов отношения
\end{itemize}
\end{frame}

\begin{frame}\frametitle{Аннотирование унификации}
\begin{itemize}
    \item $x^{Undef} \ \equiv \ [y^m \ z^n]$ $\rightarrow$ аннотация $x \ = \ max(m, n) + 1$
    \item $x^{n} \equiv [y^m \ z^{Undef} \ w^k]$ $\rightarrow$ аннотация $z \ = \ n+1$
    \item $C \ Name \ [t_0^{i_0}, \dots, t_k^{i_k}] \equiv C \ Name \ [s_0^{j_0}, \dots, s_k^{j_k}]$ $\rightarrow$ аннотируем аргументы конструкторов попарно
    \item Остальные случаи симметричны
\end{itemize}
\end{frame}

\begin{frame}[fragile]\frametitle{Аннотирование вызова}
\begin{itemize}
    \item Аннотирование самого вызова: $Undef$-аннотации аргументов равны $n \ + \ 1$, где $n$ --- максимальная аннотация аргументов
        \begin{itemize}
            \item Аннотирование невозможно, если все аргументы $Undef$ --- недостаточно информации
        \end{itemize}
    \item Аннотирование тела вызываемого отношения
        \begin{itemize}
            \item Во избежание повторного аннотирования отношения по тому же направлению, сохраним название и направление в ``стеке вызовов''
        \end{itemize}
\end{itemize}
\end{frame}

\begin{frame}[fragile]\frametitle{Несколько вызовов в дизъюнкте}
\begin{itemize}
    \item Последовательность нескольких вызовов влияет на направления их трансляции
    \item Пусть $z$ --- входная переменная
\end{itemize}
\begin{tabular}{p{3.5cm} p{3.5cm} p{3.5cm}}
    \begin{center}
\begin{lstlisting}
:: $rel^o$ x y $z^0$ =
    $f^o$ x y $\wedge$
    $h^o$ $z^0$ y $\wedge$
    $g^o$ x $z^0$
\end{lstlisting}
не достаточно информации
    \end{center}
&
    \begin{center}
\begin{lstlisting}
:: $rel^o$ x y $z^0$ =
    $h^o$ $z^0$ $y^1$ $\wedge$
    $f^o$ $x^2$ $y^1$ $\wedge$
    $g^o$ $x^1$ $z^0$
\end{lstlisting}
$x \ \coloneqq \ f \ y$
    \end{center}
&
    \begin{center}
\begin{lstlisting}
:: $rel^o$ x y $z^0$ =
    $h^o$ $z^0$ $y^1$ $\wedge$
    $g^o$ $x^1$ $z^0$ $\wedge$
    $f^o$ $x^1$ $y^1$
\end{lstlisting}
предикат $f \ x \ y$
    \end{center}
\end{tabular}
\begin{itemize}
    \item Решение: в случае неудавшегося аннотирования запустим алгоритм еще раз, изменив порядок вызовов
\end{itemize}
\end{frame}

\begin{frame}\frametitle{Терминируемость алгоритма аннотирования}
\begin{itemize}
    \item Повторное аннотирование отношений не производится $\rightarrow$ каждому отношению сопоставляется конечное количество уникальных аннотаций
    \item В случае нескольких вызовов в дизъюнкте количество перестановок вызовов конечно $\rightarrow$ конечно количество запусков алгоритма
\end{itemize}
\end{frame}

\begin{frame}\frametitle{Результаты}
\begin{itemize}
    \item Разработан алгоритм анализа времени связывания для \miniKanren{}
        \begin{itemize}
            \item он определяет порядок, в котором связываются переменные данного отношения с учётом направления его вычисления
        \end{itemize}
    \item Успешно интегрирован в транслятор
\end{itemize}
\end{frame}

\end{document}