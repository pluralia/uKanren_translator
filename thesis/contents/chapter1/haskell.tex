\subsection{Целевой язык трансляции --- подмножество языка \haskell{}}

Выбор языка \haskell{} в качестве целевого языка трансляции обоснован встроенной поддержкой ленивых вычислений, а также наличием в языке удобного механизма сопоставления с образцом. 
Эти особенности позволяют наиболее естественным образом транслировать программы на языке \miniKanren{} в функциональный язык.
Программа на \miniKanren{} способна генерировать бесконечное количество ответов, и ленивые вычисления позволяют поддержать это свойство автоматически.
Сопоставление с образцом удачно заменяет унификацию, позволяя производить присваивание сразу нескольким переменным как частям одного конструктора.

Рассматриваемое подмножество включает в себя несколько конструкций.
\begin{itemize}
    \item Сопоставление с образцом и охранные выражения для описания функций.
    \item Сопоставлениe с образцом при связывании имен в $let$-выражениях.
    \item Монада списка для моделирования недетерминизма~\cite{Wadler1985HRF} (см. раздел~\ref{sec:nedeterm} главы~\ref{translator}).
    Для большей лаконичности транслированных программ используется $do$-нотация\footnote{Описание do-нотации языка \haskell{}: \url{https://en.wikibooks.org/wiki/Haskell/do\_notation}, дата последнего посещения: 20.05.2020}.
    \item Классы типов для предоставления механизма генерации всех возможных значений подходящего типа (см. раздел~\ref{sec:transgen} главы~\ref{translator}).
\end{itemize}
