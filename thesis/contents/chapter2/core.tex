\subsection{Алгоритм аннотирования нормализованной программы}

Входные данные алгоритма: нормализованная программа на \miniKanren{} (цель и список определений) и список входных переменных.
Выходные данные: список проаннотированных определений, требуемых для вычисления цели.
Мы будем называть этот список \emph{стеком вызовов}, потому что в нем будут находиться вызываемые отношения.

\emph{Успешным результатом аннотирования} назовём ситуацию, когда получившийся по окончании выполнения алгоритм стек вызовов удовлетворяет следующим условиям:
\begin{itemize}
    \item Все отношения, требуемые для вычисления цели программы, присутствуют в стеке;
    \item Все переменные отношений, присутствующих в стеке вызовов, проаннотированы числом;
\end{itemize}

При инициализации алгоритма выполняются следующие действия: 
\begin{itemize}
    \item Все входные переменные аннотируются $0$;
    \item Создается пустой стек вызовов;
\end{itemize}

Аннотация цели осуществляется итеративно, пока не будет достигнута неподвижная точка функции, описывающей шаг аннотирования. 
За один шаг аннотируется хотя бы одна унификация или один вызов отношения.
Если в течение шага нового аннотирования не произошло, считается, что достигнута неподвижная точка.
Для аннотации цели в дищъюнктивной нормальной форме необходимо проаннотировать все её дизъюнкты. 
Аннотации переменных в дизъюнкте должны согласовываться: одна и та же переменная в конъюнктах одного дизъюнкта должна иметь одну и ту же аннотацию.
Конъюнкты аннотируются в заранее определенном порядке. 
Сначала мы аннотируем унификации, а затем вызовы отношений. 
Каждый раз при аннотации новой переменной необходимо установить ту же аннотацию всем другим вхождениям этой переменной в дизъюнкте. 

При аннотировании унификаций возможны следующие случаи. 
Здесь и далее аннотация переменной указывается в верхнем индексе.
\begin{itemize}
    \item Унификация имеет вид $x^{Undef} \equiv t[y_0^{i_0}, \dots, y_k^{i_k}]$, то есть переменная, имеющая аннотацию $Undef$, унифицируется с термом $t$ со свободными переменными $y_j^{i_j}$ с целочисленными аннотациями $i_j$. В таком случае переменной $x$ необходимо присвоить аннотацию $n + 1$, где $n = max \{ i_0, \dots i_k\}$;
    \item Переменная, аннотированная числом, унифицируется с термом: $x^{n} \equiv t[y_0^{i_0}, \dots, y_k^{i_k}]$; некоторые свободные переменные терма проаннотированны $Undef$.
    Тогда всем переменным $y_j^{Undef}$ присваивается аннотация $n+1$;
    \item Остальные случаи симметричны;
\end{itemize}

Помимо унификации конъюнкт может быть вызовом некоторого отношения.
Если переменные всех аргументов проаннотированы $Undef$, для аннотирования не достаточно информации, поэтому следует перейти к аннотации следующего конъюнкта.
Если хотя бы одна переменная-аргумент не $Undef$, произведём аннотацию вызова.
Она состоит из двух частей: аннотации аргументов самого вызова и, в случае необходимости, аннотации тела вызываемого отношения в соответствии с направлением вызова.

Aннотация тела вызываемого отношения состоит из следующих шагов:
\begin{itemize}
    \item Получение направление вызова. Для этого аннотации аргументов "сбрасываются": $Undef$ остаются таковыми, а числовые --- становятся $0$. Для вызываемого отношения не важен момент времени в прошлом, когда его входные переменные стали известны --- для него они все стали известны в момент времени $0$;
    \item Аргументы вызова подставляются вместе со "сброшенными" аннотациями в тело вызываемого отношения;
    \item Имя, направление вызова и частично проаннотированное тело помещаются в стек вызовов;
    \item Происходит запуск алгоритма аннотирования;
    \item Обновляется стек вызовов: по имени и направлению помещается тело вызова после аннотирования;
\end{itemize}

Добавление в стек вызовов информации о ранее проаннотированных в конкретных направлениях отношениях позволяет избежать повторного аннотирования.
В частности, помогает не получить бесконечный цикл при аннотировании рекурсивного вызова.
Как это происходит: если частично определенное направление текущего вызова согласовано с ранее проаннотированным, анализировать его не нужно.
Два направления назовем \emph{согласованными}, если:
\begin{itemize}
    \item Аннотации их аргументов попарно совпадают;
    \item Некоторые аннотации аргументов одного из направлений являются $Undef$, а оставшиеся, числовые, совпадают с соответствующими числами аннотаций аргументов другого направления;
\end{itemize}

Для иллюстрации понятия согласованных направлений рассмотрим следующие примеры. 
Пусть есть отношение $r^o$ с частично определенными направлениями: $r^o \ x^0 \ y^0 \ z^{Undef}$ и $r^o \ x^1 \ y^0 \ z^{Undef}$.
Они являются несогласованными, так как аннотации переменной $x$ являются числовыми и не совпадают.
При этом, направление $r^o \ x^1 \ y^0 \ z^{Undef}$ согласовано с направлением $r^o \ x^1 \ y^0 \ z^1$, так как аннотация $z$ в первом направлении является $Undef$ и, значит, может оказаться как входной, так и выходной.

Перейдём к аннотации аргументов самого вызова.
Первым шагом нужно определить, есть ли необходимость аннотировать тело вызова.
По имени и направлению проверяем наличие согласованного направления в стеке вызовов.
Если такового не оказалось, запустим аннотацию тела вызываемого отношения, а иначе сразу перейдём к аннотации аргументов.
Для этого необходимо заменить $Undef$-аннотации переменных на $n+1$, где $n$ --- максимальная аннотация переменных-аргументов вызова.
