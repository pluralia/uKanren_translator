\section{Разработка аннотатора}
\label{annotator}

Данная глава посвящена разработке аннотатора.
В первой части рассказано об адаптации идей анализа времени связывания для определения направления вычислений.
Вторая часть вводит понятие нормальной формы для программы на \miniKanren{}.
Алгоритм аннотирования описывается в третьей части.
В четвёртой части представлена корректность предложенного алгоритма.

\subsection{Анализ времени связывания}

Анализ времени связывания разделяет программные конструкции на домены согласно моментам, когда конкретная конструкция получила связывание.
В зависимости от цели применения могут выбираться разные домены времен связывания.

В данной работе цель --- указать порядок, в котором имена связываются со значениями.

\subsubsection{Обзор существующих решений}

Анализ времени связывания часто используется при offline-специализации программ~\cite{jones1993partial}. 
В этом случае он используется для определения того, какие данные известны статически и должны быть учтены при специализации, а какие неизвестны. 
Также часто определяется, какие функции вообще следует специализировать и каким образом. 

Анализ времени связывания существует для логического языка \prolog{}~\cite{leuschel2004prolog} и функционального-логического языка \mercury{}~\cite{vanhoof2004binding} --- представителей родственных реляционному программированию парадигм.
Однако, в языке \mercury{} анализ времени связывания~\cite{vanhoof2004binding} используется для эффективной компиляции. 
При этом используются только аннотации in и out --- статические и динамические переменные. 
Этого недостаточно, чтобы определить порядок вычислений при трансляции в функциональный язык.
Определение порядка вычислений в \mercury{} осуществляется во время более трудоемкого анализа модов (mode analysis), не существующего для \miniKanren{}. 
При этом непосредственное использование этого подхода для \miniKanren{} невозможно, так как не все языки семейства типизируемы, а анализ времени связывания \mercury{} осуществляется с учётом графа типов, построенного по программе. 

Система \logen{} реализует анализ времени связывания для чистого подмножества \prolog{}~\cite{leuschel2004prolog}.
Основное предназначение анализа в этой работе --- улучшение качества специализации, упорядочивания вызовов не производится. 

Работа~\cite{Thiemann1997AUF} описывает анализ времени связывания для лямбда-исчисления с функциями высшего порядка. 
Его цель также в том, чтобы определить порядок связывания переменных, поэтому авторы используют отрезок натурального ряда $\{ 0, 1, \dots, N\}$. 
Эта идея была использована в данной работе.

\subsection{Нормальная форма программы на \miniKanren{}}

Любое отношение \miniKanren{} можно преобразовать в нормальную форму.
\emph{Нормальной формой} будем называть дизъюнкцию конъюнкций вызовов отношений или унификаций термов, в которой все свободные переменные введены в область видимости в самом начале; при этом отсутствуют унификации двух конструкторов.
Соответствующий абстрактный синтаксис приведен на рисунке~\ref{fig:normmk}.

\begin{figure}
    \begin{align*}
      Goal  &: \underline{fresh} \ [Name] \ (\bigvee \bigwedge Goal') \\
      Goal' &: \underline{call} \ Name \ [Var] \\
            &\mid Var \equiv Term \\
            &\mid Term \equiv Var \\
      Term  &: Var \\ 
            &\mid \underline{cons} \ Name \ [Term]
    \end{align*}
    \caption{Абстрактный синтаксис нормализованной программы на \miniKanren{}}
    \label{fig:normmk}
\end{figure}

Приведём отличия нормализованной программы от ненормализованной.
Для каждого из них обсудим причины появления и способ получения из ненормализованной формы.
\begin{itemize}
    \item Тело определения находится в ДНФ;
    \item $fresh$-цель одна на самом верхнем уровне;
    \item Не существует унификаций термов-конструкторов;
    \item Не существует вызовов на термах-конструкторах;
\end{itemize}

Такие ограничения вводятся с целью упрощения процесса аннотирования и трансляции в целом:
\begin{itemize}
    \item ДНФ тела позволяет уменьшить глубину вложенности программы;
    \item $fresh$-цель задаёт скоуп вычислений и позволяет использовать одинаковые имена переменных в различных скоупах --- её наличие только на верхнем уровне означает, что все переменные находятся в одном скоупе и на всю программу существует только один скоуп;
    \item Отсутствие унификаций термов-конструкторов позволяет не производить очевидной унификации в процессе выполнения алгоритма;
    \item Отсутствие вызовов на термах-конструкторах позволяет избежать неопределённости в процессе аннотирования;
\end{itemize}

Если в программе на \miniKanren{} отсутствуют вызовы на термах-конструкторах, то привести её к core-\miniKanren{} несложно:
\begin{itemize}
    \item Приведение булевого выражения в ДНФ --- тривиальная задача;
    \item Если уникально переименовать все $fresh$-переменные отношения, то fresh-цель можно оставить только на самом верхнем уровне, избежав перекрытия имён;
    \item Унификацию термов-конструкторов, если совпадают их имена и количество аргументов, всегда можно заменить на унификацию переменной и терма;
    \item Способ аннотирования программ с вызовами на термах-конструкторах рассматривается в подчастях описания алгоритма;
\end{itemize}


\subsection{Алгоритм аннотирования нормализованной программы}
\label{lab:coreAnn}

Ниже описывается базовый алгоритм аннотирования нормализованной программы на \miniKanren{}, псевдокод которого представлен на рисунке~\ref{alg:annotate}.
Вспомогательные функции $annotateDisj$, $annotateUnification$ и $annotateInvoke$ приведены на рисунках~\ref{alg:annotateDisj},~\ref{alg:annotateUnification} и~\ref{alg:annotateInvoke} соответственно.

\begin{figure}[h!]
  \begin{center}
  \begin{minipage}{1.1\textwidth}
\begin{algorithm}[H]
  \KwIn{($goal$,~$scope$) --- нормализованная программа на \miniKanren{} (цель и список отношений); $inVars$ --- список входных переменных}
  \KwOut{$goal$ --- проаннотированная цель;~$stack$ --- стек вызовов}
  $stack \gets []$\;
  \For {$var~\KwFrom~goal$} {
    \eIf {$var \in inVars$} {
      $var \gets (var,~0)$
    }{
      $var \gets (var,~Undef)$
    }
  }
  \For {$disj~\KwFrom~goal$} {
    $disj \gets moveUnifsBeforeInvokes(disj)$\;
    $(disj,~stack) \gets annotateDisj(disj,~stack)$
  }
  \Return {$(goal,~stack)$}
\end{algorithm}
  \end{minipage}
  \end{center}
  \caption{Алгоритм $annotate$ для аннотирования нормализованной программы на \miniKanren{}}
  \label{alg:annotate}
\end{figure}

Алгоритм аннотирования $annotate$ получает на вход нормализованную программу на \miniKanren{} (цель и список определений), а также список входных переменных.
По окончанию его работы будут получены проаннотированная цель и ассоциативный массив, содержащий проаннотированные определения, требующихся для вычисления цели.
Ассоциативный массив представляет собой отображение пары имя-направление отношения в проаннотированную цель --- тело отношения.
Мы будем называть этот массив \emph{стеком вызовов}, потому что в нем будут находиться вызываемые отношения.

\emph{Успешным результатом аннотирования} назовём ситуацию, когда получившийся по окончании выполнения алгоритма стек вызовов удовлетворяет следующим условиям:
\begin{itemize}
    \item Все отношения, требуемые для вычисления цели программы, присутствуют в стеке;
    \item Все переменные отношений, присутствующих в стеке вызовов, проаннотированы числом.
\end{itemize}

При инициализации алгоритма выполняются следующие действия:
\begin{itemize}
    \item Все входные переменные аннотируются $0$;
    \item Создается пустой стек вызовов.
\end{itemize}

\begin{figure}[h!]
  \begin{center}
  \begin{minipage}{1.1\textwidth}
\begin{algorithm}[H]
  \KwIn{$disj$ --- дизъюнкт; $stack$ --- стек вызовов}
  \KwOut{$disj$ --- проаннотированный дизъюнкт; $stack$ --- стек вызовов}
  \While {$not(isFixedPointReached(disj,~stack))$} {
    \For {$conj~\KwFrom~disj$} {
      \Switch{$conj$} {
        \Case{$unif \gets isUnification(conj)$}{
          $(conj,~stack) \gets annotateUnification(unif)$
        }
        \Case{$invoke \gets isInvoke(conj)$}{
          $(conj,~stack) \gets annotateInvoke(invoke,~stack,~scope)$
        }
      }
      \For {$(conjVar,~conjAnn)~\KwFrom~conj$} {
        \For {$(disjVar,~disjAnn)~\KwFrom~disj$} {
          \If {$disjAnn = Undef~\KwAnd~disjVar = conjVar$} {
            $disjAnn \gets conjAnn$
          }
        }
      }
    }
  }
  \Return {$(disj,~stack)$}
\end{algorithm}
  \end{minipage}
  \end{center}
  \caption{Алгоритм $annotateDisj$ для аннотирования дизъюнкта}
  \label{alg:annotateDisj}
\end{figure}

Для аннотации цели в ДНФ необходимо проаннотировать все её дизъюнкты.
Аннотация дизъюнкта $annotateDisj$ (см. рисунок~\ref{alg:annotateDisj}) осуществляется итеративно, пока не будет достигнута неподвижная точка кода, описывающего шаг аннотирования.
За один шаг аннотируется хотя бы одна конъюнкция (унификация или вызов отношения).
Если в течение шага ни одна новая переменная не была проаннотирована, считается, что достигнута неподвижная точка.

Конъюнкты аннотируются в заранее определенном порядке: cначала мы аннотируем унификации, а затем вызовы отношений.
Данный порядок задает функция $moveUnifsBeforeInvokes$ на рисунке~\ref{alg:annotate}.
Аннотации переменных в дизъюнкте должны согласовываться: одна и та же переменная в конъюнктах одного дизъюнкта должна иметь одну и ту же аннотацию.
Каждый раз при аннотации новой переменной необходимо установить ту же аннотацию всем другим вхождениям этой переменной в дизъюнкте.

Для того, чтобы аннотировать конъюнкцию необходимо аннотировать все ее конъюнкты, то есть унификации и вызовы отношения. 
Об этом будет рассказано в следующих разделах. 
%%%%%%%%%%%%%%%%%%%%%%%%%%%%%%%%%%%%%%%%%%%%%%%%%%%%%%%%%%%%%%%%%%%%%%%%%%%%%%%%%%%%%%%%%%%%%%%%%%%%%%%%%%%%%%%%%%%%%%%

\subsubsection{Алгоритм аннотирования унификаций}

Псевдокод алгоритма аннотирования унификаций представлен на рисунке~\ref{alg:annotateUnification}.

При аннотировании унификаций возможны следующие случаи (здесь и далее аннотация переменной указывается в верхнем индексе).
\begin{itemize}
    \item Унификация имеет вид $x^{Undef} \equiv t[y_0^{i_0}, \dots, y_k^{i_k}]$, то есть переменная, имеющая аннотацию $Undef$, унифицируется с термом $t$ со свободными переменными $y_j^{i_j}$ с целочисленными аннотациями $i_j$. В таком случае переменной $x$ необходимо присвоить аннотацию $n + 1$, где $n = max \{ i_0, \dots i_k\}$ (в псевдокоде на рисунке~\ref{alg:annotateUnification} --- функция $getMaxAnnotation$).
    \item Переменная, аннотированная числом, унифицируется с термом: $x^{n} \equiv t[y_0^{i_0}, \dots, y_k^{i_k}]$; некоторые свободные переменные терма проаннотированны $Undef$.
    Тогда всем переменным $y_j^{Undef}$ присваивается аннотация $n+1$ при помощи функции $replaceUndefWith$.
    \item Остальные случаи симметричны.
\end{itemize}

\begin{figure}[h!]
  \begin{center}
  \begin{minipage}{1\textwidth}
\begin{algorithm}[H]
  \KwIn{$unif$ --- унификация}
  \KwOut{$unif$ --- унификация}
  $(left,~right) \gets unif$\;
  \Switch{$(left,~right)$} {
    \Case{$((var,~ann) \gets isUndefVariable(left),~\_)$} {
      $ann \gets getMaxAnnotation(right) + 1$
    }
    \Case{$((var,~ann) \gets isVariable(left),~\_)$} {
      $right \gets replaceUndefWith(ann + 1,~right)$
    }
    \Other{
      $//~symmetric~cases$\;
      $\dots$
    }
  }
  \Return {$unif$}
\end{algorithm}
  \end{minipage}
  \end{center}
  \caption{Алгоритм $annotateUnification$ для аннотирования унификации}
  \label{alg:annotateUnification}
\end{figure}

%%%%%%%%%%%%%%%%%%%%%%%%%%%%%%%%%%%%%%%%%%%%%%%%%%%%%%%%%%%%%%%%%%%%%%%%%%%%%%%%%%%%%%%%%%%%%%%%%%%%%%%%%%%%%%%%%%%%%%%

\subsubsection{Аннотирование вызовов отношений}

Аннотирование вызовов отношения состоит из двух частей:
\begin{itemize}
    \item аннотирования тела вызываемого отношения в соответствии с направлением вызова (опционально);
    \item аннотирования аргументов самого вызова отношения.
\end{itemize}
Псевдокод алгоритма приведен на рисунке~\ref{alg:annotateInvoke}.

\begin{figure}[h!]
  \begin{center}
  \begin{minipage}{1\textwidth}
\begin{algorithm}[H]
  \KwIn{$invoke$ --- вызов отношения; $stack$ --- стек вызовов; $scope$ --- список определений}
  \KwOut{$invoke$ --- вызов отношения; $stack$ --- стек вызовов}
  $(name,~terms) \gets invoke$\;
  $invokeDirection \gets makeInvokeDirection(terms)$\;
  $stackKey \gets (name,~invokeDirection)$\;
  \If {$NameDirectionAreNotInStack(stackKey,~stack)$} {
    $inVars \gets []$\;
    \For {$(var,~ann) \gets invokeDirection$} {
      \If {$ann~=~0$} {
        $inVars \gets var~:~inVars$
      }
    }
    $body \gets getBodyByName(name,~scope)$\;
    $stack \gets insert(stack,~stackKey,~null)$\;
    $body \gets annotation(body,~inVars)$\;
    $program \gets (body,~scope)$\;
    $(body,~stack) \gets annotate(program,~inVars)$\;
    $stack \gets insert(stack,~stackKey,~body)$
  }
  $terms \gets replaceUndefWith(getMaxAnnotation(terms) + 1,~terms)$\;
  \Return {$(invoke,~stack)$}
\end{algorithm}
  \end{minipage}
  \end{center}
  \caption{Алгоритм $annotateInvoke$ для аннотирования вызова отношения}
  \label{alg:annotateInvoke}
\end{figure}

Запускать алгоритм аннотирования тела вызываемого отношения нужно только в случае, если это ещё не было сделано для данного направления.
Чтобы определить необходимость аннотирования тела вызова, по имени вызова и его направлению проверим наличие согласованного направления в стеке вызовов.
Два направления назовем \emph{согласованными}, если аннотации их аргументов попарно равны.
Если согласованного направления не нашлось, запустим аннотирование тела вызываемого отношения.

Получим направление вызова.
Для этого аннотации аргументов обнуляются: числовые аннотации становятся $0$, а $Undef$ --- $1$.
Для вызываемого отношения не важен момент времени в прошлом, когда его входные переменные стали известны --- для него они все стали известны в момент времени $0$.
В то же время по возвращении из вызова все $Undef$ переменные станут известны --- для вызывающего отношения это следующий момент за моментом вызова.

Aннотирование тела вызываемого отношения состоит из следующих шагов:
\begin{itemize}
    \item получение входных переменных по направлению вызова;
    \item получение тела вызываемого отношения из списка определений программы при помощи функции $getBodyByName$;
    \item вставки имени и направления в стек вызовов (однако, соответствующее им тело отношение отсутствует: оно будет проаннотировано на следующем шаге и будет добавлено в стек вызовов позже);
    \item запуск алгоритма аннотирования $annotate$ (см. рисунок~\ref{alg:annotate}) для тела вызываемого отношения на обновлённом стеке вызовов;
    \item обновление стека вызовов: по имени и направлению в стек вызовов помещается тело после аннотирования.
\end{itemize}

Добавление в стек вызовов информации о ранее проаннотированных в конкретных направлениях отношениях позволяет избежать повторного аннотирования.
В частности, помогает не получить бесконечный цикл при аннотировании рекурсивного вызова.

Для аннотирования аргументов вызова отношения необходимо заменить $Undef$-аннотации переменных на $n+1$, где $n$ --- максимальная аннотация переменных-аргументов вызова.
В псевдокоде на рисунке~\ref{alg:annotateInvoke} для этого используются функции $replaceUndefWith$ и $getMaxAnnotation$.
Это верно, потому что после завершения вызова мы считаем, что все $Undef$-переменные стали известны из вызываемого отношения.
При этом, так как при аннотировании дизъюнкта сначала аннотируются все унификации, а затем --- все вызовы отношений, можно утверждать, что, к моменту аннотирования первого по порядку вызова отношения будут известны все возможные переменные.
Случай нескольких вызовов отношений в одном дизъюнкте рассматривается дополнительно в разделе~\ref{lab:disjPerm}.

\subsection{Примеры аннотирования}

В этом разделе приведено несколько примеров аннотирования отношений.
Числа над переменными обозначают аннотации.

%%%%%%%%%%%%%%%%%%%%%%%%%%%%%%%%%%%%%%%%%%%%%%%%%%%%%%%%%%%%%%%%%%%%%%%%%%%%%%%%

\subsubsection{Отношение $append^o$ в прямом направлении}

$append^o$ --- отношение связывающее три списка, первые два из которых являются конкатенацией третьего.
Его аннотирование в прямом направлении представлено на рисунке~\ref{lst:appendoIIOANN}.

В~данном случае  переменные $x$ и $y$ являются входными. 
При начале работы алгоритма, таких отношения и направления нет в стеке вызовов, поэтому добавим их и запустим рекурсивно аннотирование цели $append^o$.
Так как $x$ и $y$ --- входные переменные, их аннотации нам известны и равны $0$.

Рассмотрим аннотирование первого дизъюнкта.
$x$ и $y$ известны --- остаётся определить $z$.
Аннотация $z$ равна $1$, так как $z$ унифицируется с $y$, аннотация которой --- $0$.

Во втором дизъюнкте аннотации $h$ и $t$ в строке~\ref{line:appendoIIOANN4} можно установить, так как известна аннотация $x$.
Аннотация $h$ распространяется на~\ref{line:appendoIIOANN5} строку, а аннотация $t$ --- на~\ref{line:appendoIIOANN6} строку.
Рекурсивный вызов отношения в строке ~\ref{line:appendoIIOANN6} согласован с имеющимся в стеке, поэтому можно проаннотировать переменную $r$.
Распространяем аннотацию $r$ в строке~\ref{line:appendoIIOANN5}.
На последнем шаге аннотируем $z$ в строке~\ref{line:appendoIIOANN4}.

\begin{figure}[h!]
  \begin{center}
  \begin{minipage}{0.4\textwidth}
  \begin{lstlisting}[language=Haskell, frame=single, numbers=left,numberstyle=\small, firstnumber=1, escapechar=|]
  $append^o$ $x^0$ $y^0$ $z^1$ =
    ($x^0$ $\equiv$ [] $\wedge$ $y^0$ $\equiv$ $z^1$) $\vee$ |\label{line:appendoIIOANN2}|
    (fresh [h, t, r] (
        $x^0$ $\equiv$ $h^1$ : $t^1$ $\wedge$ |\label{line:appendoIIOANN4}|
        $z^3$ $\equiv$ $h^1$ : $r^2$ $\wedge$ |\label{line:appendoIIOANN5}|
        $append^o$ $t^1$ $y^0$ $r^2$ |\label{line:appendoIIOANN6}|
    ))
    \end{lstlisting}
  \end{minipage}
  \end{center}
  \caption{Результат аннотирования отношения $append^o$ в прямом направлении}
  \label{lst:appendoIIOANN}
\end{figure}

%%%%%%%%%%%%%%%%%%%%%%%%%%%%%%%%%%%%%%%%%%%%%%%%%%%%%%%%%%%%%%%%%%%%%%%%%%%%%%%%

\subsubsection{Отношение $append^o$ в обратном направлении}

Теперь рассмотрим аннотирование $append^o$ в обратном направлении.
В~этом случае мы считаем переменную $z$ входной (см. рисунок~\ref{lst:appendoOOIANN}).
Пусть $append^o$ уже в стеке и $z$ проаннотирована.
В первом дизъюнкте $x$ и $y$ имеют аннотацию~$1$: $y$ унифицируется со входной переменной $z$, а $x$ --- с константой.
Во втором дизъюнкте на первом шаге становятся известны аннотации $h$ и $r$ (строка~\ref{line:appendoOOIANN5}).
Аннотация $r$ распространяется на строку~\ref{line:appendoOOIANN6}. 
Отношение с согласованным направлением есть в стеке, поэтому можно аннотировать $t$ и $y$.
Далее аннотация $t$ распространяется на строку~\ref{line:appendoOOIANN4}, и на последнем шаге аннотируется $x$. 

\begin{figure}[h!]
  \begin{center}
  \begin{minipage}{0.4\textwidth}
  \begin{lstlisting}[language=Haskell, frame=single, numbers=left,numberstyle=\small, firstnumber=8, escapechar=|]
  $append^o$ $x^1$ $y^1$ $z^0$ =
    ($x^1$ $\equiv$ [] $\wedge$ $y^1$ $\equiv$ $z^0$) $\vee$ |\label{line:appendoOOIANN2}|
    (fresh [h, t, r] (
        $x^3$ $\equiv$ $h^1$ : $t^2$ $\wedge$ |\label{line:appendoOOIANN4}|
        $z^0$ $\equiv$ $h^1$ : $r^1$ $\wedge$ |\label{line:appendoOOIANN5}|
        $append^o$ $t^2$ $y^2$ $r^1$ |\label{line:appendoOOIANN6}|
    ))
    \end{lstlisting}
  \end{minipage}
  \end{center}
  \caption{Результат аннотирования отношения $append^o$ в обратном направлении}
  \label{lst:appendoOOIANN}
\end{figure}

%%%%%%%%%%%%%%%%%%%%%%%%%%%%%%%%%%%%%%%%%%%%%%%%%%%%%%%%%%%%%%%%%%%%%%%%%%%%%%%%

\subsubsection{Отношение $revers^o$ в обратном направлении}

Ещё один пример --- отношение $revers^o$.
Оно связывает два списка, получающиеся переворачиванием друг друга.
Его определение приведено в листинге~\ref{lst:reversoOIANN}.

Добавим $revers^o$ по обратному направлению в стек вызовов и проинициализируем $y$ как входную переменную.
Рассмотрим второй дизъюнкт.
На первом шаге можно попытаться проаннотировать только вызов $append^o$ в строке~\ref{line:reversoOIANN6} --- известна $y$.
Такого отношения в стеке вызовов нет --- добавляем и вызываем аннотирование.
Это и есть вызов $append^o$ в обратном направлении, рассмотренный выше (см. рисунок~\ref{lst:appendoOOIANN}).
Аннотирование $append^o$ позволяет определить аннотации переменных $r$ и $h$ --- распространяем их по другим конъюнктам.
На следующем шаге вычисляем аннотацию переменной $t$ рекурсивного вызова $revers^o$, так как он уже есть в стеке (см. строку~\ref{line:reversoOIANN6}).
Распространяем аннотацию $t$ и аннотируем $x$ на следующем шаге в строке~\ref{line:reversoOIANN4}.

\begin{figure}[h!]
  \begin{center}
  \begin{minipage}{0.4\textwidth}
  \begin{lstlisting}[language=Haskell, frame=single, numbers=left,numberstyle=\small, firstnumber=15, escapechar=|]
  $revers^o$ $x^1$ $y^0$ =
    ($x^1$ $\equiv$ [] $\wedge$ $y^0$ $\equiv$ []) $\vee$ |\label{line:reversoOIANN2}|
    (fresh [h, t, r] (
        $x^5$ $\equiv$ $h^2$ : $t^4$ $\wedge$ |\label{line:reversoOIANN4}|
        $append^o$ $r^3$ $[h^2]$ $y^0$ |\label{line:reversoOIANN5}|
        $revers^o$ $t^4$ $r^3$ $\wedge$ |\label{line:reversoOIANN6}|
    ))
    \end{lstlisting}
  \end{minipage}
  \end{center}
  \caption{Результат аннотирования отношения $revers^o$ в обратном направлении}
  \label{lst:reversoOIANN}
\end{figure}

\subsection{Нормализация программ для аннотирования}

%%%%%%%%%%%%%%%%%%%%%%%%%%%%%%%%%%%%%%%%%%%%%%%%%%%%%%%%%%%%%%%%%%%%%%%%%%%%%%%%

\subsubsection{Нерекурсивные вызовы на конструкторах}

К моменту вызова аргумент-конструктор может быть проаннотирован частично.
В этом случае неизвестно является ли переменная, соответствующая данному аргументу, входной или выходной.
Другими словами, невозможно определить направление вызова.

Для решения данной проблемы будем действовать следующим образом:
\begin{itemize}
    \item Сформируем новое отношение, принимающее на вход все переменные аргументов вызова. Его тело --- тело вызываемого отношения с подставленными в него аргументами.
    \item Вызов старого отношения на аргументах-конструкторах заменим на вызов нового отношения на аргументах-переменных. 
\end{itemize}

Рассмотрим вызов $append^o~(a : as)~ys~z$.
Один из его аргументов --- конструктор списка.
Сформируем новое отношение $append^o1$ (см. рисунок ~\ref{lst:appendo1}), осуществив подстановку $x~\rightarrow~(a~:~as)$ в тело $append^o$.
Заметим, что первый дизъюнкт $append^o$ отсутствует в $append^o1$.
Он стал заведомо ошибочен: унификация $x \equiv []$ обратилась в $(a : as) \equiv []$.
Во втором дизъюнкте первый конъюнкт обратился в унификацию двух конструкторов и, как следствие, разбился на две унификации.

\begin{figure}[h!]
  \begin{center}
  \begin{minipage}{0.4\textwidth}
  \begin{lstlisting}[language=Haskell, frame=single, numbers=left,numberstyle=\small, firstnumber=41, escapechar=|]
  $append^o1$ $a$ $as$ $y$ $z$ =
    (fresh [h, t, r] (
        a $\equiv$ h $\wedge$ |\label{line:appendo13}|
        as $\equiv$ t $\wedge$ |\label{line:appendo14}|
        z $\equiv$ h : r $\wedge$ |\label{line:appendo15}|
        $append^o$ t y r |\label{line:appendo16}|
    ))
    \end{lstlisting}
  \end{minipage}
  \end{center}
  \caption{Отнощение $append^o1$, полученное подстановкой $x~\rightarrow~(a~:~as)$ в $append^o$}
  \label{lst:appendo1}
\end{figure}

Производить замену вызова на аргументах-конструкторах нужно также в теле созданного отношения, поэтому данный алгоритм должен запускаться до достижения неподвижной точки.
В связи с этим алгоритм может зацикливаться при работе с рекурсивными вызовами на конструкторах.

%%%%%%%%%%%%%%%%%%%%%%%%%%%%%%%%%%%%%%%%%%%%%%%%%%%%%%%%%%%%%%%%%%%%%%%%%%%%%%%%

\subsubsection{Рекурсивные вызовы на конструкторах}

Рассмотрим проблему на примере.
Отношение $revacc^o$ связывает три списка: третий получается переворачиванием первого, а второй является аккумулятором.
$revacc^o$ приведено на рисунке~\ref{lst:revacco}.

\begin{figure}[h!]
  \begin{center}
  \begin{minipage}{0.4\textwidth}
  \begin{lstlisting}[language=Haskell, frame=single, numbers=left,numberstyle=\small, firstnumber=48, escapechar=|]
  $revacc^o$ $xs$ $acc$ $sx$ =
    ($xs$ $\equiv$ [] $\wedge$ $sx$ $\equiv$ $acc$) $\vee$ |\label{line:revacco2}|
    (fresh [h, t] (
        $xs$ $\equiv$ $h$ : $t$ $\wedge$ |\label{line:revacco4}|
        $revacc^o$ $t$ $(h~\%~acc)$ $sx$ |\label{line:revacco5}|
    ))
    \end{lstlisting}
  \end{minipage}
  \end{center}
  \caption{Отношение $revacc^o$}
  \label{lst:revacco}
\end{figure}

Данное отношение содержит рекурсивный вызов на конструкторе в строке~\ref{line:revacco5}.
Попробуем заменить его на новое отношение по алгоритму, описанному в предыдущей секции.
Подстановка $aсс~\rightarrow~(h~:~acc)$ в $revacc^o$ представлена на рисунке~\ref{lst:revacco1}.
На данном рисунке видно, что в строке~\ref{line:revacco15} такая подстановка привела к большей вложенности конструкторов.
Это означает, что неподвижная точка не будет достигнута никогда.

\begin{figure}[h!]
  \begin{center}
  \begin{minipage}{0.4\textwidth}
  \begin{lstlisting}[language=Haskell, frame=single, numbers=left,numberstyle=\small, firstnumber=48, escapechar=|]
  $revacc^o1$ $xs$ $h$ $acc$ $sx$ =
    ($xs$ $\equiv$ [] $\wedge$ $sx$ $\equiv$ $(h~\%~acc)$) $\vee$ |\label{line:revacco12}|
    (fresh [h', t] (
        $xs$ $\equiv$ $h'$ : $t$ $\wedge$ |\label{line:revacco14}|
        $revacc^o$ $t$ $(h'~\%~(h~\%~acc))$ $sx$ |\label{line:revacco15}|
    ))
    \end{lstlisting}
  \end{minipage}
  \end{center}
  \caption{Отношение $revacc^o1$, полученное подстановкой $acc~\rightarrow~(h~:~acc)$ в $revacc^o$}
  \label{lst:revacco1}
\end{figure}

Альтернативное решение состоит из двух шагов:
\begin{itemize}
    \item В дизъюнкт, содержащий рекурсивный вызов на конструкторе, добавим конъюнкт --- унификацию этого конструктора с новой переменной;
    \item В вызове аргумент-конструктор заменим на новую переменную.
\end{itemize}

На рисунке~\ref{lst:revacco2IOOANN} приведён пример применения данного решения и результат аннотирования $revacc^o$ в прямом направлении.

\begin{figure}[h!]
  \begin{center}
  \begin{minipage}{0.4\textwidth}
  \begin{lstlisting}[language=Haskell, frame=single, numbers=left,numberstyle=\small, firstnumber=48, escapechar=|]
  $revacc^o2$ $xs0$ $acc1$ $sx1$ =
    ($xs0$ $\equiv$ [] $\wedge$
    $sx1$ $\equiv$ $<gen:>$ $\wedge$|\label{line:revacco2IOOANN2}|
    $sx1$ $\equiv$ $acc2$) $\vee$ |\label{line:revacco2IOOANN3}|
    (fresh [h, t, hacc] (
        $xs0$ $\equiv$ $h1$ : $t1$ $\wedge$ |\label{line:revacco2IOOANN5}|
        $hacc2$ $\equiv$ $h1$ : $acc3$ $\wedge$ |\label{line:revacco2IOOANN6}|
        $revacc^o$ $t1$ $hacc2$ $sx2$ |\label{line:revacco2IOOANN7}|
    ))
    \end{lstlisting}
  \end{minipage}
  \end{center}
  \caption{Результат аннтирования отношения $revacc^o2$, полученного унификацией аргумента-конструктора с первым входным аргументом}
  \label{lst:revacco2IOOANN}
\end{figure}

Унификация позволит определять к моменту вызова, является ли аргумент, бывший конструктором, входным или выходным.
``Минусом'' данного подхода является возможность потерять информацию об аннотациях переменных конструктора для аннотирования тела вызова.
Потеря этой информации может привести к неуспешному завершению аннотирования.
Применение алгоритма генерация способно исправить ситуацию.

Данный подход может работать и для нерекурсивных вызовов на конструкторах, но он с большей вероятностью потребует генерацию, применения которой хочется избежать.

%%%%%%%%%%%%%%%%%%%%%%%%%%%%%%%%%%%%%%%%%%%%%%%%%%%%%%%%%%%%%%%%%%%%%%%%%%%%%%%%

\subsubsection{Вызовы на одних и тех же переменных}

Вызовы отношений могут происходить на одних и тех же переменных.
В этом случае аннотации соответствующих аргументов обязаны совпадать.
Это делает не валидными некоторые направления, которые, судя по количеству аргументов, должны существовать.

Рассмотрим пример: $append^o~x~x~z$.
У $append^o$ три аргумента и, значит, восемь направлений.
Однако, первые два аргумента данного вызова совпадают и направлений остаётся четыре, так как направления с разной аннотацией первых двух аргументов становятся невалидными.

Справиться с данной проблемой помогает тот же подход, что и для нерекурсивных вызовах на конструкторах: создадим новое отношение, подставив аргументы в тело исходного.
В созданном отношении (см. рисунок~\ref{lst:appendo2}) заведомо не может существовать невалидных направлений.
\begin{figure}[h!]
  \begin{center}
  \begin{minipage}{0.4\textwidth}
  \begin{lstlisting}[language=Haskell, frame=single, numbers=left,numberstyle=\small, firstnumber=57, escapechar=|]
  $append^o2$ x z =
    (x $\equiv$ [] $\wedge$ x $\equiv$ z) $\vee$ |\label{line:appendo22}|
    (fresh [h, t, r] (
        x $\equiv$ h : t $\wedge$ |\label{line:appendo24}|
        z $\equiv$ h : r $\wedge$ |\label{line:appendo25}|
        $append^o$ t x r |\label{line:appendo26}|
    ))
    \end{lstlisting}
  \end{minipage}
  \end{center}
  \caption{$append^o2$, полученное подстановкой $y~\rightarrow~x$ в $append^o$}
  \label{lst:appendo2}
\end{figure}

\subsection{Корректность алгоритма аннотирования}

Алгоритм аннотирования, представленный в работе, способен аннотировать только нормализованные программы на \miniKanren{}.
Однако, любую программу на \miniKanren{} можно привести в нормальную форму описанными выше методами~\ref{lab:normProof}.
Доказав корректность аннотирования нормализованных программ, мы докажем и корректность ненормализованных.

Алгоритм представляет собой адаптацию алгоритма анализа времени связывания для \miniKanren{}.
Для доказательства корректности необходимо показать его терминируемость и согласованность, что и сделано в последующих разделах.

%%%%%%%%%%%%%%%%%%%%%%%%%%%%%%%%%%%%%%%%%%%%%%%%%%%%%%%%%%%%%%%%%%%%%%%%%%%%%%%%

\subsubsection{Терминируемость}

\begin{theorem}
Предложенный в разделе~\ref{lab:coreAnn} алгоритм аннотирования нормализованных программ терминируется
\end{theorem}

\begin{proof}
Алгоритм терминируется, так как повторное аннотирование отношений не производится.
Это, в свою очередь, означает, что будет достигнута неподвижная точка и алгоритм завершит свою работу.
Проверим, что каждая из частей алгоритма не производит повторного аннотирования.

\textit{Унификация}.
Аннотирование происходит только в случае существования $Undef$-аннотаций.

\textit{Вызов отношения}.
Имеющиеся в стеке вызовов отношения не аннотируются снова, а в каждом отношении используется конечное количество уникальных переменных.

\textit{Конъюнкция / Дизъюнкт}.
Так как конъюнкт --- либо унификация, либо вызов отношения, то ни один конъюнкт не будет проаннотирован повторно.
Конъюнктов в дизъюнкте конечное количество, значит, неподвижная точка будет достигнута.

\textit{Объявление новых переменных}.
Только на верхнем уровне; аннотирование такой цели равнозначно аннотированию цели внутри $fresh$-цели.

\textit{Дизъюнкция}.
Аннотирование происходит независимо для каждого дизъюнкта.

Из представленых выше рассуждений можно сделать вывод, что каждому отношению можно сопоставить конечное количество уникальных аннотаций и терминируемость этой части алгоритма доказана.
\end{proof}

\begin{theorem} Предложенная в разделе~\ref{lab:disjPerm} модификация алгоритма аннотирования нормализованных программ для разрешения проблемы нескольких вызовов в одном дизъюнкте терминируется.
\end{theorem}

\begin{proof}
Существование нескольких вызовов в одном дизъюнкте приводит к необходимости применять алгоритм аннотирования ко всем возможным версиям дизъюнкта, каждая из которых отличается очередной перестановкой вызовов.
Терминируемость в этом случае следует из двух фактов:
\begin{itemize}
    \item количество перестановок вызовов конечно и, значит, конечно количество версий дизъюнкта;
    \item алгоритм аннотирования нормализованных программ терминируется.
\end{itemize}
\end{proof}

\begin{theorem} 
Предложенная в разделе~\ref{lab:gen} модификация алгоритма аннотирования нормализованных программ для разрешения проблемы зависимости $fresh$-переменных только друг от друга терминируется.
\end{theorem}

\begin{proof}
Добавление генерации не влияет на терминируемость несмотря на итеративность процесса.
Терминируемость следует из двух фактов.
\begin{itemize}
    \item Количество переменных, оставшихся в случае неуспешного аннотирования помеченными $Undef$, конечно для всего стека вызовов, так как конечно количество переменных в любом отношении, а, значит, и в стеке вызовов; на каждой итерации генерации происходит добавление хотя бы одной генерации хотя бы в одно определение со стека вызовов.
    \item Aлгоритм аннотирования нормализованных программ, запускаемый повторно после генерации, терминируется.
\end{itemize}
\end{proof}

%%%%%%%%%%%%%%%%%%%%%%%%%%%%%%%%%%%%%%%%%%%%%%%%%%%%%%%%%%%%%%%%%%%%%%%%%%%%%%%%

\subsubsection{Согласованность}

В анализе времени связывания под согласованностью понимается \emph{зависимость статических данных только от статических}: статические данные не могут определяться динамическими.

\begin{theorem} 
Предложенный в разделе~\ref{lab:coreAnn} алгоритм аннотирования нормализованных программ с модификациями для разрешения проблемы нескольких вызовов в одном дизъюнкте~\ref{lab:disjPerm} и проблемы зависимости $fresh$-переменных только друг от друга~\ref{lab:gen} является согласованным.
\end{theorem}

\begin{proof}
Вычисление дизъюнктов в \miniKanren{} происходит независимо, значит, и аннотировать их можно независимо.
Как следствие, для аннотации тела отношения необходимо проаннотировать входящие в него дизъюнкты.
Показав корректность аннотирования одного дизъюнкта, покажем корректность аннотирования всего тела.

Каждый дизъюнкт --- это конъюнкция вызовов и унификаций.
Вычисление конъюнктов в \miniKanren{} происходит одновременно: значение полученное в одном конъюнкте, мгновенно становится известно в другом.
Для аннотирования это означает, что, если стала известна аннотация целевой переменной в одном конъюнкте, она мгновенно становится известна во всех конъюнктах, в которые эта переменная входит.
Именно так и происходит в алгоритме: дизъюнкты аннотируются независимо, а аннотация переменной, ставшая известной в одном конъюнкте, распространяется на все вхождения этой переменной в другие конъюнкты.

Введём понятие зависимости одной переменной от другой в рамках предложенного алгоритма.
Понятие отношения подразумевает ``равноправие'' переменных, участвующих в нём.
Однако, при выборе конкретного направления вычисления значения переменных множества $X$ становятся известны раньше значений переменных множества $Y$.
В этом случае будем говорить, что переменные Y \emph{зависят} от переменных X.
Проиллюстрируем понятие зависимости переменных друг от друга на примерах.

Пример: зависимость для унификаций.
Пусть есть два конъюнкта: $x \equiv y$ и $y \equiv 7$.
Во втором конъюнкте $7$ --- константа, поэтому мы можем проунифицировать $y$ и сказать, что $y = 7$.
В этот же момент мы узнаем в первом конъюнкте, что $y$ стала известна, и можем превратить унификацию в равенство $x = y$, обозначающее зависимость $x$ от $y$.

Пример: зависимость для вызовов отношений.
Пусть есть вызов отношения $append^o~x~y~z$, где мы уже знаем из других конъюнктов значение $z$.
В этом случае алгоритм посчитает, что этот вызов $append^o$ происходит направлении, при котором переменные $x$ и $y$ являются выходными.
В этом случае можно говорить о зависимости $x$ и $y$ от $z$: $(x,~y)~=~append^o~z$ (в случае недетерминированной семантики $apppend^o$ корректнее говорить о $[(x,~y)]~=~append^o~z$).

Введём инвариант, отражающий идею согласованности.
Доказав его выполнение на любом шаге алгоритма, докажем его корректность.

\emph{Инвариант:
\begin{itemize}
    \item либо переменная не проаннотирована (имеет аннотацию $Undef$);
    \item либо переменная проаннотирована числом; тогда существует хотя бы один конъюнкт, в котором все переменные, от которых она зависит, проаннотированы строго меньшими числами.
\end{itemize}
}

Рассмотрим алгоритм, чтобы убедиться в выполнении инварианта.
В начальный момент времени аннотацию $0$ имеют только входные переменные.
Остальные переменные проаннотированы $Undef$.

Конъюнкты отсортированы: вызовы следуют за унификациями.
К каждой унификации применяется алгоритм аннотирования унификаций, в точности выполняющий инвариант.
$Undef$-аннотация целевой переменной заменяется всегда на строго большее значение, чем значение аннотации любой переменной, от которой целевая переменная зависит.
После аннотирования каждого конъюнкта информация об аннотациях его переменных распространяется на все оставшиеся конъюнкты.
Следующий для аннотирования конъюнкт обладает релевантными аннотациями.

К моменту начала аннотирования вызовов отношения можем быть уверены, что в текущем вызове известны все аннотации переменных, которые можно было получить из унификаций.
Все другие --- только из последующих вызовов отношений.
Тем самым, мы знаем направление первого вызова в текущей перестановке вызовов конкретного дизъюнкта.
При наличии нескольких вызовов их порядок влияет на аннотирование.
Наилучший порядок, позволяющий получить проаннотированное отношение, можно найти только опытным путём --- перебрав все перестановки вызовов.
Поэтому, без ограничения общности можно считать, что первый вызов выбран верно.
Если аннотирование при этом закончится неудачей, запустится аннотирование того же дизъюнкта с другим порядком вызовов.
Важно заметить, что, в случае неуспеха аннотирования стек вызовов будет содержать переменные с $Undef$ аннотациями --- это является частью инварианта.

Вернёмся к аннотированию вызова.
Алгоритм аннотации аргументов вызова в точности соблюдает инвариант.
Каждое вызываемое в конкретном направлении отношение добавляется в стек, если оно там отсутствовало, и инициализируется так, что его входные переменные имеют аннотацию $0$.
Это позволяет рассматривать аннотацию тела вызываемого отношения независимо от причин аннотирования: является ли аннотируемая цель целью программы или телом вызываемого отношения.

Для случая необходимости добавления генерации остаётся заметить, что данная модификация лишь изменяют структуру программы, но не производит аннотирование.
За счёт чего можно утверждать, что инвариант сохраняется.
\end{proof}

