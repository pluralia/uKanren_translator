\subsection{Трансляция в функциональный язык}

Суть трансляции:
\emph{По отношению с фиксированным направлением генерируется функция на функциональном языке программирования}.

Существуют трансляторы логических программ, но все они обладают спецификой \prolog{} или стараются сохранить полимодальность при трансляции.
Так, в~\cite{Matsushita1997FCO} обсуждается проблема трансляции $cut$-операции.
\cite{Bellia1986TRB} рассматривают способы интеграции логического и функционального программирования.
Демонстрируется невозможность трансляции унификации в сопоставление с образцом по причине возможности унификации вычисляться в различных направлениях.
В случае транслятора, предлагаемого в данной работе, направление быть должно и вышеупомянутой проблемы нет.
Ту же проблему имеет \cite{marchiori1995the} --- попытка сохранить полимодальность при трансляции.
