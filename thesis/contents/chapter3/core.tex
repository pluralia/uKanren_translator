\subsection{Алгоритм аннотирования нормализованной программы}
\label{lab:coreAnn}

Ниже описывается базовый алгоритм аннотирования нормализованной программы на \miniKanren{}, псевдокод которого представлен на рисунке~\ref{alg:annotate}.
Вспомогательные функции $annotateDisj$, $annotateUnification$ и $annotateInvoke$ приведены на рисунках~\ref{alg:annotateDisj},~\ref{alg:annotateUnification} и~\ref{alg:annotateInvoke} соответственно.

\begin{figure}[h!]
  \begin{center}
  \begin{minipage}{1.1\textwidth}
\begin{algorithm}[H]
  \KwIn{($goal$,~$scope$) --- нормализованная программа на \miniKanren{} (цель и список отношений); $inVars$ --- список входных переменных}
  \KwOut{$goal$ --- проаннотированная цель;~$stack$ --- стек вызовов}
  $stack \gets []$\;
  \For {$var~\KwFrom~goal$} {
    \eIf {$var \in inVars$} {
      $var \gets (var,~0)$
    }{
      $var \gets (var,~Undef)$
    }
  }
  \For {$disj~\KwFrom~goal$} {
    $disj \gets moveUnifsBeforeInvokes(disj)$\;
    $(disj,~stack) \gets annotateDisj(disj,~stack)$
  }
  \Return {$(goal,~stack)$}
\end{algorithm}
  \end{minipage}
  \end{center}
  \caption{Алгоритм $annotate$ для аннотирования нормализованной программы на \miniKanren{}}
  \label{alg:annotate}
\end{figure}

Алгоритм аннотирования $annotate$ получает на вход нормализованную программу на \miniKanren{} (цель и список определений), а также список входных переменных.
По окончанию его работы будут получены проаннотированная цель и ассоциативный массив, содержащий проаннотированные определения, требующихся для вычисления цели.
Ассоциативный массив представляет собой отображение пары имя-направление отношения в проаннотированную цель --- тело отношения.
Мы будем называть этот массив \emph{стеком вызовов}, потому что в нем будут находиться вызываемые отношения.

\emph{Успешным результатом аннотирования} назовём ситуацию, когда получившийся по окончании выполнения алгоритма стек вызовов удовлетворяет следующим условиям:
\begin{itemize}
    \item Все отношения, требуемые для вычисления цели программы, присутствуют в стеке;
    \item Все переменные отношений, присутствующих в стеке вызовов, проаннотированы числом.
\end{itemize}

При инициализации алгоритма выполняются следующие действия:
\begin{itemize}
    \item Все входные переменные аннотируются $0$;
    \item Создается пустой стек вызовов.
\end{itemize}

\begin{figure}[h!]
  \begin{center}
  \begin{minipage}{1.1\textwidth}
\begin{algorithm}[H]
  \KwIn{$disj$ --- дизъюнкт; $stack$ --- стек вызовов}
  \KwOut{$disj$ --- проаннотированный дизъюнкт; $stack$ --- стек вызовов}
  \While {$not(isFixedPointReached(disj,~stack))$} {
    \For {$conj~\KwFrom~disj$} {
      \Switch{$conj$} {
        \Case{$unif \gets isUnification(conj)$}{
          $(conj,~stack) \gets annotateUnification(unif)$
        }
        \Case{$invoke \gets isInvoke(conj)$}{
          $(conj,~stack) \gets annotateInvoke(invoke,~stack,~scope)$
        }
      }
      \For {$(conjVar,~conjAnn)~\KwFrom~conj$} {
        \For {$(disjVar,~disjAnn)~\KwFrom~disj$} {
          \If {$disjAnn = Undef~\KwAnd~disjVar = conjVar$} {
            $disjAnn \gets conjAnn$
          }
        }
      }
    }
  }
  \Return {$(disj,~stack)$}
\end{algorithm}
  \end{minipage}
  \end{center}
  \caption{Алгоритм $annotateDisj$ для аннотирования дизъюнкта}
  \label{alg:annotateDisj}
\end{figure}

Для аннотации цели в ДНФ необходимо проаннотировать все её дизъюнкты.
Аннотация дизъюнкта $annotateDisj$ (см. рисунок~\ref{alg:annotateDisj}) осуществляется итеративно, пока не будет достигнута неподвижная точка кода, описывающего шаг аннотирования.
За один шаг аннотируется хотя бы одна конъюнкция (унификация или вызов отношения).
Если в течение шага ни одна новая переменная не была проаннотирована, считается, что достигнута неподвижная точка.

Конъюнкты аннотируются в заранее определенном порядке: cначала мы аннотируем унификации, а затем вызовы отношений.
Данный порядок задает функция $moveUnifsBeforeInvokes$ на рисунке~\ref{alg:annotate}.
Аннотации переменных в дизъюнкте должны согласовываться: одна и та же переменная в конъюнктах одного дизъюнкта должна иметь одну и ту же аннотацию.
Каждый раз при аннотации новой переменной необходимо установить ту же аннотацию всем другим вхождениям этой переменной в дизъюнкте.

Для того, чтобы аннотировать конъюнкцию необходимо аннотировать все ее конъюнкты, то есть унификации и вызовы отношения. 
Об этом будет рассказано в следующих разделах. 
%%%%%%%%%%%%%%%%%%%%%%%%%%%%%%%%%%%%%%%%%%%%%%%%%%%%%%%%%%%%%%%%%%%%%%%%%%%%%%%%%%%%%%%%%%%%%%%%%%%%%%%%%%%%%%%%%%%%%%%

\subsubsection{Алгоритм аннотирования унификаций}

Псевдокод алгоритма аннотирования унификаций представлен на рисунке~\ref{alg:annotateUnification}.

При аннотировании унификаций возможны следующие случаи (здесь и далее аннотация переменной указывается в верхнем индексе).
\begin{itemize}
    \item Унификация имеет вид $x^{Undef} \equiv t[y_0^{i_0}, \dots, y_k^{i_k}]$, то есть переменная, имеющая аннотацию $Undef$, унифицируется с термом $t$ со свободными переменными $y_j^{i_j}$ с целочисленными аннотациями $i_j$. В таком случае переменной $x$ необходимо присвоить аннотацию $n + 1$, где $n = max \{ i_0, \dots i_k\}$ (в псевдокоде на рисунке~\ref{alg:annotateUnification} --- функция $getMaxAnnotation$).
    \item Переменная, аннотированная числом, унифицируется с термом: $x^{n} \equiv t[y_0^{i_0}, \dots, y_k^{i_k}]$; некоторые свободные переменные терма проаннотированны $Undef$.
    Тогда всем переменным $y_j^{Undef}$ присваивается аннотация $n+1$ при помощи функции $replaceUndefWith$.
    \item Остальные случаи симметричны.
\end{itemize}

\begin{figure}[h!]
  \begin{center}
  \begin{minipage}{1\textwidth}
\begin{algorithm}[H]
  \KwIn{$unif$ --- унификация}
  \KwOut{$unif$ --- унификация}
  $(left,~right) \gets unif$\;
  \Switch{$(left,~right)$} {
    \Case{$((var,~ann) \gets isUndefVariable(left),~\_)$} {
      $ann \gets getMaxAnnotation(right) + 1$
    }
    \Case{$((var,~ann) \gets isVariable(left),~\_)$} {
      $right \gets replaceUndefWith(ann + 1,~right)$
    }
    \Other{
      $//~symmetric~cases$\;
      $\dots$
    }
  }
  \Return {$unif$}
\end{algorithm}
  \end{minipage}
  \end{center}
  \caption{Алгоритм $annotateUnification$ для аннотирования унификации}
  \label{alg:annotateUnification}
\end{figure}

%%%%%%%%%%%%%%%%%%%%%%%%%%%%%%%%%%%%%%%%%%%%%%%%%%%%%%%%%%%%%%%%%%%%%%%%%%%%%%%%%%%%%%%%%%%%%%%%%%%%%%%%%%%%%%%%%%%%%%%

\subsubsection{Аннотирование вызовов отношений}

Аннотирование вызовов отношения состоит из двух частей:
\begin{itemize}
    \item аннотирования тела вызываемого отношения в соответствии с направлением вызова (опционально);
    \item аннотирования аргументов самого вызова отношения.
\end{itemize}
Псевдокод алгоритма приведен на рисунке~\ref{alg:annotateInvoke}.

\begin{figure}[h!]
  \begin{center}
  \begin{minipage}{1\textwidth}
\begin{algorithm}[H]
  \KwIn{$invoke$ --- вызов отношения; $stack$ --- стек вызовов; $scope$ --- список определений}
  \KwOut{$invoke$ --- вызов отношения; $stack$ --- стек вызовов}
  $(name,~terms) \gets invoke$\;
  $invokeDirection \gets makeInvokeDirection(terms)$\;
  $stackKey \gets (name,~invokeDirection)$\;
  \If {$NameDirectionAreNotInStack(stackKey,~stack)$} {
    $inVars \gets []$\;
    \For {$(var,~ann) \gets invokeDirection$} {
      \If {$ann~=~0$} {
        $inVars \gets var~:~inVars$
      }
    }
    $body \gets getBodyByName(name,~scope)$\;
    $stack \gets insert(stack,~stackKey,~null)$\;
    $body \gets annotation(body,~inVars)$\;
    $program \gets (body,~scope)$\;
    $(body,~stack) \gets annotate(program,~inVars)$\;
    $stack \gets insert(stack,~stackKey,~body)$
  }
  $terms \gets replaceUndefWith(getMaxAnnotation(terms) + 1,~terms)$\;
  \Return {$(invoke,~stack)$}
\end{algorithm}
  \end{minipage}
  \end{center}
  \caption{Алгоритм $annotateInvoke$ для аннотирования вызова отношения}
  \label{alg:annotateInvoke}
\end{figure}

Запускать алгоритм аннотирования тела вызываемого отношения нужно только в случае, если это ещё не было сделано для данного направления.
Чтобы определить необходимость аннотирования тела вызова, по имени вызова и его направлению проверим наличие согласованного направления в стеке вызовов.
Два направления назовем \emph{согласованными}, если аннотации их аргументов попарно равны.
Если согласованного направления не нашлось, запустим аннотирование тела вызываемого отношения.

Получим направление вызова.
Для этого аннотации аргументов обнуляются: числовые аннотации становятся $0$, а $Undef$ --- $1$.
Для вызываемого отношения не важен момент времени в прошлом, когда его входные переменные стали известны --- для него они все стали известны в момент времени $0$.
В то же время по возвращении из вызова все $Undef$ переменные станут известны --- для вызывающего отношения это следующий момент за моментом вызова.

Aннотирование тела вызываемого отношения состоит из следующих шагов:
\begin{itemize}
    \item получение входных переменных по направлению вызова;
    \item получение тела вызываемого отношения из списка определений программы при помощи функции $getBodyByName$;
    \item вставки имени и направления в стек вызовов (однако, соответствующее им тело отношение отсутствует: оно будет проаннотировано на следующем шаге и будет добавлено в стек вызовов позже);
    \item запуск алгоритма аннотирования $annotate$ (см. рисунок~\ref{alg:annotate}) для тела вызываемого отношения на обновлённом стеке вызовов;
    \item обновление стека вызовов: по имени и направлению в стек вызовов помещается тело после аннотирования.
\end{itemize}

Добавление в стек вызовов информации о ранее проаннотированных в конкретных направлениях отношениях позволяет избежать повторного аннотирования.
В частности, помогает не получить бесконечный цикл при аннотировании рекурсивного вызова.

Для аннотирования аргументов вызова отношения необходимо заменить $Undef$-аннотации переменных на $n+1$, где $n$ --- максимальная аннотация переменных-аргументов вызова.
В псевдокоде на рисунке~\ref{alg:annotateInvoke} для этого используются функции $replaceUndefWith$ и $getMaxAnnotation$.
Это верно, потому что после завершения вызова мы считаем, что все $Undef$-переменные стали известны из вызываемого отношения.
При этом, так как при аннотировании дизъюнкта сначала аннотируются все унификации, а затем --- все вызовы отношений, можно утверждать, что, к моменту аннотирования первого по порядку вызова отношения будут известны все возможные переменные.
Случай нескольких вызовов отношений в одном дизъюнкте рассматривается дополнительно в разделе~\ref{lab:disjPerm}.
