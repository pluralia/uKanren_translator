\subsection{Корректность алгоритма}

Трансляция считается корректной, если семантика полученной после трансляции функции совпадает с семантикой исходного отношения в заданном направлении.
В целом, для этого необходимо перебрать все программы на \miniKanren{}, применить к ним алгоритм трансляции и проверить, изменилось ли семантика.
Однако, перебор все программ --- алгоритмически неразрешимая задача.

Для доказательства корректности предлагается сделать следующее:
\begin{itemize}
    \item Рассмотреть корректность перенесения особенностей \miniKanren{} в функциональную парадигму;
    \item Проанализировать семантику конструктов \miniKanren{} в терминах функциональной парадигмы;
    \item Проверить, что конструкты функционального языка, полученные из конструктов \miniKanren{}, обладают той же семантикой;
\end{itemize}

Такой подход не гарантирует полностью корректный транслятор (что сделать невозможно), но позволит говорить о его корректности в первом приближении.

В разделе "Особенности \miniKanren{} и способы их трансляции" рассмотрены способы транслировать особенности \miniKanren{} в функциональную парадигму.
Корректность этих способов вытекает из их определения.

Проанализируем, чем являются конструкты нормализованной программы на \miniKanren{} для функциональной парадигмы:
\begin{itemize}
    \item Fresh --- объявление и определение переменной;
    \item Унификация или вызов отношения на полностью известных переменных --- предикат, проверка соответствия переменных;
    \item Унификация --- переопределение переменной: считаем, что при fresh переменная объявляется и получает своё значение --- все возможные значения её типа;
    \item Вызов отношения --- вызов функции, возвращает результат;
    \item Конъюнкт --- переопределение или предикат;
    \item Дизъюнкт --- функция, скоуп: внутри конъюнкции все конъюнкты, представляющие собой переопределения, могут использовать определённые ими переменные;
    \item Тело отношения --- запуск нескольких функций и объединение их результатов;
\end{itemize}

Проанализируем результат трансляции для каждого их этих конструктов и проверим сохранение семантики:
\begin{itemize}
    \item Унификация или вызов отношения на всех известных переменных --- $if$ или $guard$;
    \item Fresh + Унификация --- сопоставление с образом или определение;
    \item Fresh + Вызов отношения --- определение;
    \item Конъюнкт --- определение, $if$ или $guard$;
    \item Дизъюнкт --- вспомогательная функция;
    \item Тело отношения --- конкатенация результатов вызовов вспомогательных функций;
\end{itemize}

Таким образом мы показали, что для описанных конструктов семантика сохраняется.
Однако, мы не рассмотрели случай вызовов отношений, где все аргументы являются выходными.
Такие программы запрещены для трансляции и на текущий момент являются ограничением подхода.