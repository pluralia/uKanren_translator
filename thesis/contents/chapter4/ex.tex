\subsection{Примеры аннотирования}

В этом разделе приведено несколько примеров аннотирования отношений.
Числа над переменными обозначают аннотации.

%%%%%%%%%%%%%%%%%%%%%%%%%%%%%%%%%%%%%%%%%%%%%%%%%%%%%%%%%%%%%%%%%%%%%%%%%%%%%%%%

\subsubsection{Отношение $append^o$ в прямом направлении}

$append^o$ --- отношение связывающее три списка, первые два из которых являются конкатенацией третьего.
Его аннотирование в прямом направлении представлено на рисунке~\ref{lst:appendoIIOANN}.

В~данном случае  переменные $x$ и $y$ являются входными. 
При начале работы алгоритма, таких отношения и направления нет в стеке вызовов, поэтому добавим их и запустим рекурсивно аннотирование цели $append^o$.
Так как $x$ и $y$ --- входные переменные, их аннотации нам известны и равны $0$.

Рассмотрим аннотирование первого дизъюнкта.
$x$ и $y$ известны --- остаётся определить $z$.
Аннотация $z$ равна $1$, так как $z$ унифицируется с $y$, аннотация которой --- $0$.

Во втором дизъюнкте аннотации $h$ и $t$ в строке~\ref{line:appendoIIOANN4} можно установить, так как известна аннотация $x$.
Аннотация $h$ распространяется на~\ref{line:appendoIIOANN5} строку, а аннотация $t$ --- на~\ref{line:appendoIIOANN6} строку.
Рекурсивный вызов отношения в строке ~\ref{line:appendoIIOANN6} согласован с имеющимся в стеке, поэтому можно проаннотировать переменную $r$.
Распространяем аннотацию $r$ в строке~\ref{line:appendoIIOANN5}.
На последнем шаге аннотируем $z$ в строке~\ref{line:appendoIIOANN4}.

\begin{figure}[h!]
  \begin{center}
  \begin{minipage}{0.4\textwidth}
  \begin{lstlisting}[language=Haskell, frame=single, numbers=left,numberstyle=\small, firstnumber=1, escapechar=|]
  $append^o$ $x^0$ $y^0$ $z^1$ =
    ($x^0$ $\equiv$ [] $\wedge$ $y^0$ $\equiv$ $z^1$) $\vee$ |\label{line:appendoIIOANN2}|
    (fresh [h, t, r] (
        $x^0$ $\equiv$ $h^1$ : $t^1$ $\wedge$ |\label{line:appendoIIOANN4}|
        $z^3$ $\equiv$ $h^1$ : $r^2$ $\wedge$ |\label{line:appendoIIOANN5}|
        $append^o$ $t^1$ $y^0$ $r^2$ |\label{line:appendoIIOANN6}|
    ))
    \end{lstlisting}
  \end{minipage}
  \end{center}
  \caption{Результат аннотирования отношения $append^o$ в прямом направлении}
  \label{lst:appendoIIOANN}
\end{figure}

%%%%%%%%%%%%%%%%%%%%%%%%%%%%%%%%%%%%%%%%%%%%%%%%%%%%%%%%%%%%%%%%%%%%%%%%%%%%%%%%

\subsubsection{Отношение $append^o$ в обратном направлении}

Теперь рассмотрим аннотирование $append^o$ в обратном направлении.
В~этом случае мы считаем переменную $z$ входной (см. рисунок~\ref{lst:appendoOOIANN}).
Пусть $append^o$ уже в стеке и $z$ проаннотирована.
В первом дизъюнкте $x$ и $y$ имеют аннотацию~$1$: $y$ унифицируется со входной переменной $z$, а $x$ --- с константой.
Во втором дизъюнкте на первом шаге становятся известны аннотации $h$ и $r$ (строка~\ref{line:appendoOOIANN5}).
Аннотация $r$ распространяется на строку~\ref{line:appendoOOIANN6}. 
Отношение с согласованным направлением есть в стеке, поэтому можно аннотировать $t$ и $y$.
Далее аннотация $t$ распространяется на строку~\ref{line:appendoOOIANN4}, и на последнем шаге аннотируется $x$. 

\begin{figure}[h!]
  \begin{center}
  \begin{minipage}{0.4\textwidth}
  \begin{lstlisting}[language=Haskell, frame=single, numbers=left,numberstyle=\small, firstnumber=8, escapechar=|]
  $append^o$ $x^1$ $y^1$ $z^0$ =
    ($x^1$ $\equiv$ [] $\wedge$ $y^1$ $\equiv$ $z^0$) $\vee$ |\label{line:appendoOOIANN2}|
    (fresh [h, t, r] (
        $x^3$ $\equiv$ $h^1$ : $t^2$ $\wedge$ |\label{line:appendoOOIANN4}|
        $z^0$ $\equiv$ $h^1$ : $r^1$ $\wedge$ |\label{line:appendoOOIANN5}|
        $append^o$ $t^2$ $y^2$ $r^1$ |\label{line:appendoOOIANN6}|
    ))
    \end{lstlisting}
  \end{minipage}
  \end{center}
  \caption{Результат аннотирования отношения $append^o$ в обратном направлении}
  \label{lst:appendoOOIANN}
\end{figure}

%%%%%%%%%%%%%%%%%%%%%%%%%%%%%%%%%%%%%%%%%%%%%%%%%%%%%%%%%%%%%%%%%%%%%%%%%%%%%%%%

\subsubsection{Отношение $revers^o$ в обратном направлении}

Ещё один пример --- отношение $revers^o$.
Оно связывает два списка, получающиеся переворачиванием друг друга.
Его определение приведено в листинге~\ref{lst:reversoOIANN}.

Добавим $revers^o$ по обратному направлению в стек вызовов и проинициализируем $y$ как входную переменную.
Рассмотрим второй дизъюнкт.
На первом шаге можно попытаться проаннотировать только вызов $append^o$ в строке~\ref{line:reversoOIANN6} --- известна $y$.
Такого отношения в стеке вызовов нет --- добавляем и вызываем аннотирование.
Это и есть вызов $append^o$ в обратном направлении, рассмотренный выше (см. рисунок~\ref{lst:appendoOOIANN}).
Аннотирование $append^o$ позволяет определить аннотации переменных $r$ и $h$ --- распространяем их по другим конъюнктам.
На следующем шаге вычисляем аннотацию переменной $t$ рекурсивного вызова $revers^o$, так как он уже есть в стеке (см. строку~\ref{line:reversoOIANN6}).
Распространяем аннотацию $t$ и аннотируем $x$ на следующем шаге в строке~\ref{line:reversoOIANN4}.

\begin{figure}[h!]
  \begin{center}
  \begin{minipage}{0.4\textwidth}
  \begin{lstlisting}[language=Haskell, frame=single, numbers=left,numberstyle=\small, firstnumber=15, escapechar=|]
  $revers^o$ $x^1$ $y^0$ =
    ($x^1$ $\equiv$ [] $\wedge$ $y^0$ $\equiv$ []) $\vee$ |\label{line:reversoOIANN2}|
    (fresh [h, t, r] (
        $x^5$ $\equiv$ $h^2$ : $t^4$ $\wedge$ |\label{line:reversoOIANN4}|
        $append^o$ $r^3$ $[h^2]$ $y^0$ |\label{line:reversoOIANN5}|
        $revers^o$ $t^4$ $r^3$ $\wedge$ |\label{line:reversoOIANN6}|
    ))
    \end{lstlisting}
  \end{minipage}
  \end{center}
  \caption{Результат аннотирования отношения $revers^o$ в обратном направлении}
  \label{lst:reversoOIANN}
\end{figure}
