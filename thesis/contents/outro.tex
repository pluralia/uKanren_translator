\section*{Заключение}

\paragraph{Целью} данной работы было создание транслятора реляционного языка в функциональный, способного в транслированной функции сохранить семантику исходного отношения в выбранном направлении.

Для достижения этой цели было поставлено несколько задач, каждая из которых была решена.

\begin{itemize}
    \item Разработка алгоритма аннотирования, позволяющего транслятору определять направления и порядок вычислений конъюнктов.
    
    Для программ на \miniKanren{} введено понятие нормальной формы. На основе идеи анализа времени связывания разработан алгоритм аннотирования переменных для нормализованных программ. Рассмотрены способы приведения любых программ в нормальную форму как часть алгоритма аннотирования. Его реализация написана на \haskell{}. Доказана корректность.

    \item Разработка алгоритма транслирования абстрактного синтаксиса реляционного языка в абстрактный синтаксис функционального языка.

    Рассмотрены особенности \miniKanren{} и особенности трансляции. С их учётом создан абстрактный синтаксис функционального языка и разработан алгоритм трансляции. Реализация написана на \haskell{}. Доказана корректность.
    
    \item Тестирование разработанного инструмента и анализ результатов.
    
    Предложено несколько классификаций программ на \miniKanren{} в соответствии с проблемами, возникшими при создании алгоритмов аннотирования и трансляции. Создана база программ на \miniKanren{}, покрывающая предложенные классификации. Для тестирования результата трансляции в конкретном синтаксисе создан конкретный синтаксис \miniKanren{} и реализован его парсер, а так же транслятор абстрактного функционального синтаксиса в конкретный. Оба алгоритма реализованы на \haskell{}. Проанализированы причины невозможности трансляции некоторых программ.
    
\end{itemize}